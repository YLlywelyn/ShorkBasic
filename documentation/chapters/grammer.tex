%!TeX program = XeLatex
%!TeX root = ../Shork Sharp.tex
\chapter{Grammar}

This is a notation for writing down the grammar of the language.  It uses regex syntax, with the components themselves being italicised.

\begin{tabular}{|r|l|}
\hline
\textit{statements} & NEWLINE* \textit{statement} (NEWLINE+ \textit{statement})* NEWLINE* \\\hline

\multirow{4}{*}{\textit{statement}} & KEYWORD:RETURN \textit{expression}? \\
                                    & KEYWORD:CONTINUE \\
                                    & KEYWORD:BREAK \\
                                    & \textit{expression} \\\hline
                                    
\multirow{2}{*}{\textit{expression}} & KEYWORD:VAR IDENTIFIER = \textit{expression} \\
                                     & \makecell{\textit{comparision\_expression} ((KEYWORD:AND | KEYWORD:OR)\\ \textit{comparision\_expression})*} \\\hline

\multirow{2}{*}{\textit{comparision\_expression}} & KEYWORD:NOT \textit{comparision\_expression} \\
                                                  & \textit{arithmatic\_expression} ((==|!=|<|<=|>|>=) \textit{arithmatic\_expression})* \\\hline

\textit{arithmatic\_expression} & \textit{term} ((+|-) \textit{term})* \\\hline

\textit{term} & \textit{factor} ((\textbackslash*|/) \textit{factor})* \\\hline

\multirow{2}{*}{\textit{factor}} & (+|-)? \textit{factor} \\
                                 & \textit{exponent} \\\hline

\textit{exponent} & \textit{call} (\^{} \textit{factor})* \\\hline
\end{tabular}
